% !TeX root = RJwrapper.tex
\title{cvcqv: A Package for Robust Estimation of Relative Variability}
\author{by Maani Beigy}

\maketitle

\abstract{
Coefficient of variation (\emph{cv}) and coefficient of quartile variation (\emph{cqv}) are widely used measures of relative dispersion which play descriptive and inferential roles (e.g., reliability analysis, quality control, inequality measurement, and anomaly detection) in various scientific fields such as biological and medical sciences, economics, actuarial sciences, laboratory sciences, etc. Since \emph{cv} and \emph{cqv} are unit-free, they are useful for comparing data from different distributions, data measured using different scales, or widely different means. However, to avoid common misuses of them, confidence intervals (\emph{CI}) are necessary. The \textbf{cvcqv} package provides a home for such tools. To our knowledge, the new R package \textbf{cvcqv} is the first R implementation of \textbf{cqv} as a robust measure of variability along with almost all available methods for the calculation of \emph{CI} for \emph{cv} and \emph{cqv}. This paper demonstrates this functionality using reproducible examples on real datasets.
}

\section{Section title in sentence case}

Introductory section which may include references in parentheses
\citep{R}, or cite a reference such as \citet{R} in the text.

\section{Another section}

This section may contain a figure such as Figure~\ref{figure:rlogo}.

\begin{figure}[htbp]
  \centering
  \includegraphics{Rlogo-5}
  \caption{The logo of R.}
  \label{figure:rlogo}
\end{figure}

\section{Another section}

There will likely be several sections, perhaps including code snippets, such as:

\begin{example}
  x <- 1:10
  result <- myFunction(x)
\end{example}

\section{Summary}

This file is only a basic article template. For full details of \emph{The R Journal} style and information on how to prepare your article for submission, see the \href{https://journal.r-project.org/share/author-guide.pdf}{Instructions for Authors}.

\bibliography{RJreferences}

\address{Author One\\
  Affiliation\\
  Address\\
  Country\\
  (ORCiD if desired)\\
  \email{author1@work}}

\address{Author Two\\
  Affiliation\\
  Address\\
  Country\\
  (ORCiD if desired)\\
  \email{author2@work}}

\address{Author Three\\
  Affiliation\\
  Address\\
  Country\\
  (ORCiD if desired)\\
  \email{author3@work}}
